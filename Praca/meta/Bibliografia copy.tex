\bibliographystyle{plain}                       % styl bibliografii
\begin{thebibliography}{3}                      % pocz�tek �rodowiska
\addcontentsline{toc}{chapter}{Wykaz literatury}    % dodaje bibliografi� do spisu tre�ci
\small              

    \bibitem{dactyloc}
Kristian Cujia, Martin Wirz, Mikkel Baun Kjrergaard, Daniel Roggen and Gerhard Troster: \emph{DactyLoc: A minimally geo-referenced WiFi+GSM-fingerprint-based localization method for positioning in urban spaces}, International Conference on Indoor Positioning and Indoor Navigation, 2012 
    \bibitem{Duda}      % \bibitem{etykieta}
Duda A.: \emph{Wprowadzenie do topologii}, PWN, Warszawa 1986
% nast�pna pozycja
    \bibitem{EngeSiek}
Engelking R., Sieklucki K.: \emph{Geometria i topologia. Cz�� II. Topologia}, PWN, Warszawa 1980
% nast�pna pozycja
    \bibitem{Patk}
Patkowska H.: \emph{Wst�p do topologii}, PWN, Warszawa 1979
% nast�pna pozycja
    \bibitem{Siek}
Sieklucki K.: \emph{Geometria i topologia. Cz�� I. Geometria}, PWN, Warszawa 1979
% nast�pna pozycja
\bibitem{link}
National Center of Biotechnology Information, http://www.ncbi.nlm.nih.gov (data dost�pu 20.12.2012 r.).
  
\end{thebibliography}                           % koniec �rodowiska